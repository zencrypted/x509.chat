\documentclass{article}

\usepackage[utf8]{inputenc}
\usepackage[T1]{fontenc}
\usepackage{helvet}
\usepackage{geometry}
\usepackage{xcolor}
\usepackage{eso-pic}
\usepackage{fancyhdr}
\usepackage{parskip}
\usepackage{microtype}
\usepackage[english]{babel}
\usepackage{url}
\usepackage{amsmath}
\usepackage{amssymb}
\usepackage{amsthm}

\definecolor{nato}{HTML}{293757}
\definecolor{footergray}{gray}{0.65}

\geometry{a4paper,left=3cm,right=3cm,top=3cm,bottom=3cm}
\renewcommand{\familydefault}{\sfdefault}
\setcounter{secnumdepth}{-2}

\let\oldsection\section
\renewcommand\section[1]{%
  \vspace{30pt}%
  {\color{nato}\Huge\bfseries #1\par}%
  \vspace{12pt}%
  \addcontentsline{toc}{section}{#1}%
}

\let\oldsubsection\subsection
\renewcommand\subsection[1]{%
  \vspace{20pt}%
  {\color{nato}\LARGE\bfseries #1\par}%
  \vspace{8pt}%
  \addcontentsline{toc}{subsection}{#1}%
}

\pagestyle{fancy}
\fancyhf{}
\fancyhead[R]{\small\color{footergray} FEB 14, 2026}
\fancyfoot[L]{\small\color{footergray} Zen Crypted Chat X.509 Encryption Architecture}
\fancyfoot[R]{\small\color{footergray}\thepage}

\renewcommand{\headrulewidth}{0pt}
\renewcommand{\footrulewidth}{0.4pt}
\renewcommand{\footrule}{\color{footergray}\hrule width \textwidth height 0.4pt}

\begin{document}
\pagestyle{fancy}

% ==================== Cover page ====================
\begin{titlepage}
\AddToShipoutPictureBG*{\AtPageLowerLeft{\color{nato}\rule{\paperwidth}{\paperheight}}}
\color{white}
\thispagestyle{empty}

\vspace*{4cm}
\centering

% Placeholder for logo (replace with \includegraphics if you have the file)
{\LARGE \textbf{Zen Crypted} Chat X.509\par}
\vspace{2cm}

{\Huge\bfseries Tactical Communicator\par}
\vspace{0.8cm}
{\LARGE\bfseries Encryption Architecture\par}
\vspace{2.5cm}
{\large Technical White Paper\par}

\vfill

\small
Version 1.0 (11) Originally published February 14, 2026\par

\end{titlepage}
\clearpage

% ==================== Table of Contents page ====================

\begin{center}
{\Huge\color{nato}\bfseries Contents\par}
\end{center}
\vspace{40pt}
\pagestyle{fancy}

\tableofcontents

\begin{abstract}
Secure messaging applications rely on end-to-end encryption (E2EE)
to protect user communications from interception and tampering.
This article examines the cryptographic protocols employed by
five messaging systems --- Threema, Signal, Session, WhatsApp, and Chat X.509
--- focusing on their ciphers, algorithms, and underlying design decisions.
\end{abstract}

\clearpage
\pagestyle{fancy}

\section{Cryptographic Properties}

\subsection{Introduction}

End-to-end encrypted messaging has become a cornerstone of digital privacy. The strength of such systems depends critically on the cryptographic primitives and protocol constructions chosen during development. This work compares five representative secure messaging protocols with respect to their key agreement mechanisms, symmetric encryption schemes, authentication methods, secrecy properties, and supporting cryptographic libraries.

The protocols under consideration are:
\begin{itemize}
    \item Chat X.509 v1/v2
    \item Session (Oxen/Loki network)
    \item Signal Protocol (basis for Signal and WhatsApp)
    \item Threema
    \item Viber
    \item WhatsApp (modified Signal Protocol implementation)
\end{itemize}

The comparison is structured around a set of core cryptographic parameters, followed by an analysis of the design trade-offs that led to the observed differences.

\subsection{Key Agreement}

Key agreement establishes shared secrets between parties. All examined protocols use elliptic-curve-based Diffie--Hellman variants.

\subsubsection*{NIST P-256 ECDH}
The NIST P-256 curve (secp256r1) is a 256-bit elliptic curve over the prime field \(\mathbb{F}_p\) with \(p = 2^{256} - 2^{224} + 2^{192} + 2^{96} - 1\), following the short Weierstrass equation
\[
y^2 \equiv x^3 - 3x + b \pmod{p}.
\]
It provides approximately 128 bits of security with cofactor \(h=1\). Standardized in NIST FIPS 186-4, it is widely supported but has faced criticism for its parameter origins.
NIST P-256 ECDH is used in Chat X.509 v1.

\subsubsection*{Curve25519 / X25519}
Curve25519 is a Montgomery curve over \(\mathbb{F}_p\) with \(p = 2^{255} - 19\), equation
\[
By^2 = x^3 + Ax^2 + x, \quad A = 486662, \, B = 1.
\]
The X25519 function provides constant-time scalar multiplication with cofactor \(h=8\) and approximately 128-bit security. Its design prioritizes side-channel resistance and transparent parameters.
Curve25519 / X25519 is used in Threema, Signal, WhatsApp, Session, Chat X.509 v2, Viber.

\subsubsection*{Ed25519}
Ed25519 is a twisted Edwards-form elliptic curve digital signature scheme derived from Curve25519, operating over the same prime field \(\mathbb{F}_p\) with \(p = 2^{255} - 19\). The curve equation is
\[
-x^2 + y^2 = 1 + dx^2 y^2,
\]
It provides roughly 128 bits of security with cofactor \(h=8\). Ed25519 uses the Edwards birational equivalence to Curve25519 for efficient, constant-time implementations, offering strong resistance to side-channel attacks and hash-function weaknesses via its EdDSA construction (double-hash of message and private key). It achieves the highest security level among widely deployed signature schemes (against both classical and quantum side-channel threats) and is standardized in RFC 8032 with widespread adoption in protocols like SSH, Signal, Session, and TLS.

\subsubsection*{Ephemeral-Static ECDH}
A single ephemeral ECDH exchange using one party's ephemeral key and the other's
static public key establishes a session key. Provides forward secrecy for that
session but no per-message key evolution. Ephemeral-Static ECDH keys provide minimal forward secrecy
and are used in Threema, Session, Chat X.509 v1.

\subsubsection*{X3DH + Double Ratchet}
Extended Triple Diffie --- Hellman (X3DH) combines multiple ECDH exchanges (including signed
prekeys and one-time prekeys) for asynchronous authenticated key agreement. The Double Ratchet algorithm then applies
symmetric KDF ratcheting and periodic DH ratcheting to derive per-message keys,
achieving both forward secrecy and post-compromise security. X3DH is used in Signal, WhatsApp, Viber.

\subsection{Identity / Long-term Keys}

Long-term keys authenticate parties and bind identities.

\subsubsection*{Curve25519}
Pure static Curve25519 keys registered with the server; no separate signing keys.
Curve25519 DH is used in Threema.

\subsubsection*{Ed25519 + X25519}
Long-term X25519 for key agreement, separate Ed25519 key for signing prekeys and identity authentication.
Ed25519 is a twisted Edwards curve over the same field as Curve25519:
\[
-x^2 + y^2 = 1 + d x^2 y^2,
\]
with high security and constant-time implementation (RFC 8032).
Ed25519 is used in Signal, WhatsApp, Session, Viber.

\subsubsection*{X.509 certificate-bound EC keys}
Traditional X.509 certificates (ITU-T X.509 / RFC 5280) containing either
NIST P-256 (v1) or X25519 (v2) public keys, encoded in ASN.1 DER (X.690).
Certificates chain to trusted roots, enabling enterprise PKI integration.
X.509 certificate envelopes for keys are used in Chat X.509 v1/v2.

\subsection{Key Derivation}

\subsubsection*{HKDF-SHA256/512}
HMAC-based Extract-and-Expand KDF (RFC 5869) using SHA-256 or SHA-512.
Provides domain separation and strong extraction from shared secrets.
HKDF is used in Signal, WhatsApp, Chat X.509, Viber.

\subsubsection*{Argon2id}
Memory-hard password-based KDF (RFC 9106), hybrid of
data-independent (Argon2i) and data-dependent (Argon2d) memory access.
Designed to resist GPU/ASIC cracking.
Argon2id is used in Session.

\subsubsection*{HSalsa20}
Core Salsa20 function applied to a 256-bit key and 128-bit nonce
to derive a 256-bit subkey. Used in NaCl/libsodium for XSalsa20
nonce extension.
HSalsa20 is used in Threema.

\subsection{Symmetric Encryption}

\subsubsection*{AES-256}
256-bit key Advanced Encryption Standard (FIPS 197) block cipher in GCM or IGE mode.
AES-256-GCM is used in Signal, WhatsApp, Chat X.509 v1.

\subsubsection*{ChaCha20}
20-round variant of Salsa20 stream cipher (RFC 8439) with 256-bit
key and 96/128-bit nonce. Addition-rotation-XOR design offers excellent
software performance and timing-attack resistance.
ChaCha20 is used in Chat X.509 v2.

\subsubsection*{XSalsa20}
Salsa20/20 core with 192-bit nonce extension. First 128 nonce bits and
key run through HSalsa20 to produce subkey; remaining 64 bits used as
standard Salsa20 nonce. Allows safe random nonce selection.
XSalsa20 is used in Threema, Session.

\subsection{Authentication / MAC}

\subsubsection*{Poly1305}
One-time Wegman--Carter authenticator over \(\mathbb{F}_{2^{130}-5}\),
128-bit security with unique keys/nonces.
Poly1305 is used in Threema, Session, Chat X.509 v2.

\subsubsection*{GCM tag}
Galois-field authentication (GHASH) providing 128-bit security (birthday bound).
GCM tag is used in Signal, WhatsApp, Chat X.509 v1, Viber.

\subsubsection*{HMAC-SHA256}
Standard HMAC construction for additional authentication in ratchet chains.
HMAC is used in Signal additional layers.

\subsection{Authenticated Encryption}

\subsubsection*{AES-256-GCM}
Counter-mode encryption with Galois/Counter Mode authentication.
Parallelizable, hardware-accelerated via AES-NI, ~128-bit security.
AES-256-GCM AE is used in Signal, WhatsApp, Chat X.509 v1, Viber.

\subsubsection*{ChaCha20-Poly1305}
RFC 8439 AEAD: ChaCha20 keystream XOR encryption + Poly1305 authentication.
High software speed, mandatory in TLS 1.3.
ChaCha20-Poly1305 is used in Chat X.509 v2

\subsubsection*{XSalsa20-Poly1305}
NaCl-style separate encryption and authentication (not strictly AEAD but
equivalent security when composed correctly). libsodium sealed boxes:
XSalsa20 encryption + Poly1305 authentication using derived one-time keys.
XSalsa20-Poly1305 is used in Session and Threema.

\subsection{Forward Secrecy}

\subsubsection*{Double Ratchet}
Per-message key deletion and DH ratcheting ensure past messages remain
confidential even if long-term keys are later compromised.
Double Ratchet is used in Signal, WhatsApp, Viber.

\subsubsection*{Ephemeral}
Session keys derived from ephemeral ECDH; provides FS for the session duration but not per-message.
Ephemeral without ratchet is used in Threema, Session, Chat X.509.

\subsection{Post-Compromise Security}

\subsubsection*{Double Ratchet}
Asymmetric DH ratchet introduces fresh entropy, allowing recovery of
confidentiality after temporary state compromise.
Double Ratchet is used in Signal, WhatsApp, Viber.

\subsubsection*{Ratchet healing}
Custom mechanism providing limited post-compromise recovery (implementation details not fully standardized).
Ratchet healing is used in Chat X.509 v2.

\section{Encryption Schemes}


\subsection{Signal and WhatsApp}

The Signal Protocol combines the Extended Triple Diffie--Hellman (X3DH)
key agreement with the Double Ratchet for per-message key evolution.
This design provides both forward secrecy and post-compromise security
while supporting asynchronous message delivery and out-of-order message
processing.

\begin{center}
\begin{tabular}{|l|l|l|l|l|l|}
\hline
\textbf{Parameter}       & \textbf{Signal Protocol}       & \textbf{WhatsApp} \\ \hline
Key Agreement            & X25519 (X3DH + 2-Ratchet)      & X25519 (X3DH + 2-Ratchet) \\
Identity / Long-term     & X25519 + Ed25519 signing       & X25519 + Ed25519 signing \\
Key Derivation           & HKDF (SHA-256/512)             & HKDF (SHA-256/512) \\
Symmetric Encryption     & AES-256-GCM                    & AES-256-GCM \\
Authentication / MAC     & GCM tag / HMAC-SHA256          & GCM tag \\
Authenticated Encryption & AES-256-GCM                    & AES-256-GCM \\
Forward Secrecy          & Yes (Double Ratchet)           & Yes (Double Ratchet) \\
Post-Compromise Security & Yes                            & Yes \\
Standards / Format       & Custom binary                  & Custom binary (Signal) \\ \hline
\end{tabular}
\end{center}

\paragraph{}
The choice of Curve25519 reflects a preference for modern,
implementation-resistant curves over legacy NIST curves. AES-256-GCM
was selected for its hardware acceleration and misuse resistance.
WhatsApp inherits this design but modifies group key management
(Sender Keys) to scale to very large user bases.


\subsection{Chat X.509}

\begin{center}
\begin{tabular}{|l|l|l|l|l|l|}
\hline
\textbf{Parameter}       & \textbf{Chat X.509 v1}         & \textbf{Chat X.509 v2} \\ \hline
Key Agreement            & Curve25519 ECDH                & X25519 (3XDH + 2-Ratchet) \\
Identity / Long-term     & X.509 cert + X25519            & X.509 cert + X25519 \\
Key Derivation           & HKDF-SHA256                    & HKDF-SHA256 \\
Symmetric Encryption     & ChaCha20                       & ChaCha20 \\
Authentication / MAC     & 16-byte Poly1305 tag           & 16-byte Poly1305 tag \\
Authenticated Encryption & ChaCha20-Poly1305              & ChaCha20-Poly1305 \\
Forward Secrecy          & Ephemeral Static               & Yes (2-Ratchet) \\
Post-Compromise Security & No                             & Yes (Ratchet healing) \\
Standards / Format       & X.509 X.894 X.680 X.690        & X.509 X.894 X.680 X.690 \\ \hline
\end{tabular}
\end{center}

\paragraph{}
This protocol integrates traditional X.509 public-key infrastructure with
NIST P-256 and CMS enveloped data formats. The design favors interoperability
with enterprise PKI environments and standards-based tooling over modern secrecy
properties. The lack of ratcheting is a deliberate simplification,
trading advanced security for compatibility.

\newpage
\subsection{Session and Threema}

Session uses libsodium sealed boxes and Argon2id for memory-hard key derivation,
reflecting a focus on resistance to offline attacks and decentralized routing.
Like Threema, it omits ratcheting to reduce complexity and state requirements,
accepting the absence of post-compromise security.

Threema adopts the NaCl/libsodium cryptographic API, using XSalsa20-Poly1305
for encryption and Curve25519 for key agreement without ratcheting. The design
prioritizes implementation simplicity, constant-time execution, and low latency
over post-compromise recovery. This choice is reasonable for a system emphasizing
minimal server-side state and moderate group sizes (up to 256 members). \\

\begin{center}
\begin{tabular}{|l|l|l|l|l|l|}
\hline
\textbf{Parameter}       & \textbf{Threema}               & \textbf{Session} \\ \hline
Key Agreement            & Curve25519 ECDH (Ephemeral)    & Ed25519-X25519 \\
Identity / Long-term     & Curve25519 key pair            & Ed25519-X25519 \\
Key Derivation           & HSalsa20                       & Argon2id \\
Symmetric Encryption     & XSalsa20                       & XSalsa20-Poly1305 \\
Authentication / MAC     & Poly1305                       & Poly1305 \\
Authenticated Encryption & XSalsa20 + Poly1305            & XSalsa20-Poly1305 \\
Forward Secrecy          & Ephemeral (no ratchet)         & Ephemeral (no ratchet) \\
Post-Compromise Security & No                             & No \\
Standards / Format       & Custom binary (NaCl-inspired)  & libsodium sealed boxes \\ \hline
\end{tabular}
\end{center}

\subsection{Viber}

\begin{center}
\begin{tabular}{|l|l|l|}
\hline
\textbf{Parameter}       & \textbf{Telegram Secret Chats} & \textbf{Viber} \\ \hline
Key Agreement            & 2048-bit DH                    & X25519 (X3DH + 2-Ratchet) \\
Identity / Long-term     & Server-mediated (long-term)    & X25519 + Ed25519 signing \\
Key Derivation           & Custom (SHA-256 based)         & HKDF (SHA-256/512) \\
Symmetric Encryption     & AES-256-IGE                    & AES-256-GCM \\
Authentication / MAC     & Custom SHA-256 msg\_key        & GCM tag \\
Authenticated Encryption & Custom (IGE + MAC)             & AES-256-GCM \\
Forward Secrecy          & Yes (100msgs/1week)            & Yes (2-Ratchet) \\
Post-Compromise Security & No                             & Yes \\
Standards / Format       & MTProto 2.0 custom             & Proprietary (2-Ratchet) \\ \hline
\end{tabular}
\end{center}

\newpage
\subsection{Conclusion}

All five protocols currently offer low quantum resistance due to their reliance on elliptic curves vulnerable to Shor's algorithm. Hybrid post-quantum constructions (e.g., PQXDH combining X25519 with lattice-based key encapsulation mechanisms) are already being deployed experimentally in Signal-based applications. The ongoing standardization of Messaging Layer Security (MLS) promises improved group key management with forward secrecy and post-compromise security at scale.

The cryptographic design choices in secure messengers reflect different priorities: maximal secrecy properties (Signal, WhatsApp), implementation simplicity and performance (Threema, Session), or standards compliance and interoperability (Chat X.509 v1). As quantum threats mature and group messaging requirements grow, future protocols will likely combine hybrid post-quantum key exchange, ratchet-based secrecy, and MLS-style group management.

\begin{thebibliography}{9}

\bibitem{threema:crypto-whitepaper}
Threema GmbH.
\newblock Cryptography Whitepaper.
\newblock Threema GmbH, March 2025.
\newblock Version: March 13, 2025.
\newblock \url{https://threema.com/press-files/2_documentation/cryptography_whitepaper.pdf}.

\bibitem{signal:doubleratchet}
Trevor Perrin, Moxie Marlinspike, and Rolfe Schmidt.
\newblock The Double Ratchet Algorithm.
\newblock Signal Messenger, November 2025.
\newblock Revision 4, 2025-11-04.
\newblock \url{https://signal.org/docs/specifications/doubleratchet/doubleratchet.pdf}.

\bibitem{signal:x3dh}
Trevor Perrin.
\newblock The X3DH Key Agreement Protocol.
\newblock Signal Messenger, November 2016.
\newblock Revision 1, 2016-11-04.
\newblock \url{https://signal.org/docs/specifications/x3dh/x3dh.pdf}.

\bibitem{whatsapp:encryption-overview}
WhatsApp LLC.
\newblock WhatsApp Encryption Overview: Technical White Paper.
\newblock WhatsApp LLC, August 2024.
\newblock Version 8, Updated August 19, 2024.
\newblock \url{https://www.whatsapp.com/security/WhatsApp-Security-Whitepaper.pdf}.

\bibitem{jefferys:session-whitepaper}
Kee Jefferys, Maxim Shishmarev, and Simon Harman.
\newblock Session: End-To-End Encrypted Conversations With Minimal Metadata Leakage.
\newblock \emph{arXiv preprint arXiv:2002.04609}, July 2024.
\newblock Updated July 4, 2024.
\newblock \url{https://arxiv.org/pdf/2002.04609}.

\bibitem{oxen:session-whitepaper-orig}
Loki Project.
\newblock Loki Network Whitepaper.
\newblock July 2018.
\newblock Early foundation document for Session / Oxen messaging.
\newblock \url{https://loki.network/wp-content/uploads/2020/02/Whitepaper.pdf}.

\bibitem{viber:encryption-overview}
Rakuten Viber.
\newblock Viber Encryption Overview.
\newblock Technical whitepaper.
\newblock \url{https://www.viber.com/app/uploads/viber-encryption-overview.pdf}.

\end{thebibliography}

\end{document}
